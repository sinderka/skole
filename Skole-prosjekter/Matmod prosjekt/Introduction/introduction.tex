\documentclass{article}
\usepackage{amsmath}
\usepackage{bm}
\usepackage{graphicx}
\usepackage[utf8]{inputenc}
\usepackage{caption}
\usepackage{subcaption}
\title{Modeling a synaptic transmission, \\
Mathematical modeling}
\author{Student numbers }


\begin{document}
\maketitle


\section{Introduction}
%We chose project 1. Our focus was making numerical solvers for the problem in all dimensions, and putting som boundary conditions on the edges to estimate the time for the signalt to transmitt. We also modeled the time using what we have learned for matematical modeling. \\
We tried to find the time estimate for the signal to be transmitted in several different ways. 
A good way to start was by finding a time 	scale. 


In this report, we start by showing a Monte Carlo simulation of random walk as a first model for the diffusion of the neurotransmitters.  Next, modeling equations are derived for the initial model, as well as the expanded model which includes the glia cells.  After this, mathematical modeling is used to obtain a rough estimate of the time used to transmit a signal. This is followed by several attempts at solving a 1D model of the problem, and then a 2D model.



\begin{thebibliography}{9}
%\cite[p. 34]{holstad}%
\bibitem{holstad}
  Jörg Henrik Holstad,
  \emph{Modellering av Diffusjon av Nevrotransmittere
i den Ekstracellulære Væsken}.
  2011.\\
https://www.duo.uio.no/bitstream/handle/10852/10871/MasteroppgaveHenrikHolstad.pdf
Retrieved 13.11.2014
\end{thebibliography}
\end{document}