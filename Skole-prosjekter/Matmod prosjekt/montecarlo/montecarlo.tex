\documentclass{article}
\usepackage{amsmath}
\usepackage{bm}
\usepackage{graphicx}
\usepackage[utf8]{inputenc}
\usepackage{caption}
\usepackage{subcaption}
\title{Simulating a synaptic cleft using monte carlo methods}
\author{Jon Christian Halvorsen }


\begin{document}
\maketitle

\section*{Simulating a synaptic transmission using monte carlo methods}

Since this project involves solving a lot of equations, and solving equations is a lot of hard work, we'll preliminary just try our hands on doing some simulations. We're doing a monte carlo simulation of diffusion in a cubic room scaled to represent a synaptic cleft. We release 5000 neurotransmitters in the middle of one side (wall $a$) and apply a brownian motian to each of the particles, thereby watching how it unfolds.

In the simplest version we just scattered the neurotransmitters, confirming that they scatter nearly uniformly out in each 3 dimensions.

The next step would be to lock all the elements inside a box (setting the flux out to zero) and implementing the dendrite wall with receptors. We call the dendrite wall $c$, opposite end of       $a$. We then assume:
\begin{itemize}
\item The probability of a neurotransmitter binding with a receptor is dependent on the distance from the dendrite wall (wall c) and how many of the receptors are already taken.
\item The probability of a neurotransmitter disconnecting from the receptor is also quantifiable so we can get some form of equilibrium  (at least when all the walls are closed).
\end{itemize}

Implementing this is also rather straightforward. As a last step we factor in the process of clearing the cleft after the signal is sent. We assume:
\begin{itemize}
\item There exists GLIA-cells in walls $b$ and $d$.
\item The two last walls are for now free, but have no flux out of them.
\item The probability for a neurotransmitter to bind to a GLIA-cell is dependent on the distance to the cell.
\item We also have a probability of "unbonding" with the GLIA-cell, and when we are the neurotransmitter is marked as inactive.
\item All inactive neurotransmitters have a probability (dependent on distance) of going back into the axon, thus clearing the synaptic cleft one particle at the time.
\item The probability of leaving the axon as an inactive neurotransmitter is equal to zero.
\item 
\end{itemize}


\end{document}
