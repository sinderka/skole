

\section{Deriving the modeling equations}
\subsection*{Diffusion equation}
Given the equations from the task \cite{task}, we quickly conclude that the diffusion of the neurotransmitters can be modeled by
\begin{align}
\label{eq:diffusion}
\frac{dc}{dt} &= \kappa \nabla^2 c \\
\label{eq:BC}
\nabla c \cdot n &= g(t,c),
\end{align}
with the last equation being the Neumann boundary condition. In the equations, $c$ is the concentration of neurotransmitters, $\kappa$ is the diffusion constant and $g(t,c)$ is a boundary flux.


\subsection{The binding process}
First we look at the reversible chemical reaction
\begin{align}
\label{eq:boundNeurotransmitter}
\text{R} + \text{N} \rightleftharpoons \text{RN},
\end{align}
where R is the number of receptors and N is the number of neurotransmitters.
The reaction rates are $k_1$ to the right and $k_2$ to the left, being respectively the probability for the reactions to occur in their direction. 
We get 3 ODE's from this chemical reaction:
\begin{align*}
\frac{d[\text{R}]}{dt} &= -k_1[\text{R}][\text{N}] + k_2[\text{RN}]\\
\frac{d[\text{N}]}{dt} &= -k_1[\text{R}][\text{N}] + k_2[\text{RN}]\\
\frac{d[\text{RN}]}{dt} &= k_1[\text{R}][\text{N}] - k_2[\text{RN}],
\end{align*}
where [R], [N] and [RN] are the concentrations of the receptors, neurotransmitters and the bound product of them.
We may consider [N][R] the probability of a neurotransmitter meeting an unoccupied receptor, and $k_1$ the probability of the binding reaction happening. Likewise for $k_2$. Next, we insert $c$ for $[\text{N}]$.  Introducing $P^R$ as the probability of a receptor being unoccupied, and $(1-P^R)$ as the probability that the neurotransmitter is attached to the receptor leads to the following simplification \cite{holstad} 
of the above ODE's:
\begin{align}
\label{eq:changeInConcentration}
\frac{dc}{dt} &= -k_1cP^R + k_2(1-P^R)\\
\label{eq:changeInPR}
\frac{dP^R}{dt} &= -k_1cP^R + k_2(1-P^R).
\end{align}

\subsection{Glia cells}
\begin{align*}
\text{T} + \text{N} \rightleftharpoons \text{TN} \rightarrow \text{N}_{\text{inactive}} + \text{T}.
\end{align*}
Here, we define $k_3, k_4, k_5$ as the reaction rates of first rightward, first leftward, second rightward reactions.

Similarly to the binding process, we get the following set of equations:
\begin{align*}
\kappa \nabla c \cdot n &= -k_3c P^T + k_4(1-P^T)\\
\frac{dP^T}{dt} &= -k_3c P^T + (1-P^T)(k_4 + k_5).\\
\end{align*}
Combining these equations, we get
\begin{align*}
\kappa \nabla c \cdot n &= -c(k_1 P^R + k_3 P^T) + k_2 (1-P^R) + k_4 (1-P^T)    \\
\frac{dP^R}{dt} &= -ck_1 P^R + k_2(1-P^R)\\
\frac{dP^T}{dt} &= -ck_3 P^T + (k_4 + k_5)(1-P^T).\\
\end{align*}

